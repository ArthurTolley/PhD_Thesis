We have explored the current methods which have been used to directly detect hundreds of gravitational waves from compact binary coalescences using the most sensitive observatories to ever be constructed. In this thesis, the research output has focused on improving the search for gravitational wave signals with both the development of new tools and the implementation of powerful existing tools in new capacities.

First, we created a novel \scladj noise artefact model and searched for these noise artefacts, removing them and re-searching for signals to find previously missed gravitational wave injections. This method was the first of its kind, creating a glitch search pipeline using techniques similar to the gravitational wave search pipelines but with glitch specific techniques to account for properties unique to the glitch class being searched for. We demonstrated that the approach taken in the ArchEnemy pipeline did not produce a significant improvement to the search sensitivity but that it was possible to model and remove glitches to find previously missed gravitational wave injections. In the future, the feasibility of glitch search pipelines can be expanded to include many classes of glitch and a data cleaning step could be made prior to searching for gravitational waves.

We implemented a exponential noise model in the PyCBC Live search for gravitational waves in low-latency, improving the detection ranking statistic estimation of gravitational wave signals. We demonstrated that by using a noise model which is more descriptive of the PyCBC Live noise we are able to gain large increases in detection sensitivity. The PyCBC Live search still lacks ranking statistic components that have been developed for the offline gravitational wave search, in the future these can be converted and optimised for the PyCBC Live search to further improve the detection ranking statistic and significance estimation of gravitational wave signals.

Finally, we have analysed, evaluated and suggested improvements for the PyCBC Live Early Warning search to maximise the number of gravitational wave events that we can see in the early warning regime. Running the PyCBC Live Early Warning search over a gravitational wave injection mock data set of $40$ days has revealed significant deficiencies with the current PyCBC Live Early Warning ranking statistic and suggestions have been made to increase the slop of the coincidence timing window allowance and to recreate the \verb|phasetd| phase-time-amplitude histograms with a specific tuning for the early warning search to allow coincidences to be made when single detector trigger pairs are found outside the light-travel time. With these improvements the future configuration of the PyCBC Live Early Warning search will identify more gravitational wave signals in early warning and this can hopefully be demonstrated on any new binary neutron star events.

The current epoch of gravitational wave astronomy is ripe for development improvements in gravitational wave searches which do not depend on detector improvements. New techniques will power gravitational wave search sensitivity adjacently to detector sensitivity increases and these techniques have applicability to new detectors which will explore the greater gravitational wave parameter space. By improving our searches we can observe a greater number of events, and in the case of early warning, with as much warning as possible to enhance our understanding of the Universe.