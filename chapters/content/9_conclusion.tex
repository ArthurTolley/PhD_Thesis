We have explored the current methods that have been used to directly detect hundreds of gravitational waves from compact binary coalescences using the most sensitive observatories ever constructed. In this thesis, the research output has focused on improving the search for gravitational wave signals by both developing new tools and applying powerful existing tools in novel ways.

First, we created a novel \scladj noise artefact model and searched for these noise artefacts, removing them and re-searching for signals to recover previously missed gravitational wave injections. This method was the first of its kind, establishing a glitch search pipeline that employed techniques similar to gravitational wave search pipelines, but tailored to account for properties unique to specific glitch classes. We demonstrated that the approach used in the ArchEnemy pipeline did not significantly improve search sensitivity, but it did show that modelling and removing glitches could allow for the recovery of previously missed gravitational wave injections. In the future, the feasibility of glitch search pipelines could be expanded to include more classes of glitches, enabling a pre-search data cleaning step before gravitational wave detection.

We implemented an exponential noise model in the PyCBC Live search for gravitational waves in low-latency, enhancing the detection ranking statistic used to evaluate gravitational wave signals. By using a noise model that more accurately describes the characteristics of PyCBC Live’s noise, we demonstrated significant gains in detection sensitivity. While PyCBC Live lacks some ranking statistic components that have been developed for offline gravitational wave searches, future work could convert and optimise these components for PyCBC Live to further improve both the detection ranking statistic and the significance estimation of gravitational wave signals.

Then, we analysed, evaluated, and proposed improvements for the PyCBC Live Early Warning search to maximise the number of gravitational wave events that can be detected in the early warning regime. Running the PyCBC Live Early Warning search over a 40-day mock data set of gravitational wave injections revealed significant deficiencies in the current ranking statistic. Based on this analysis, we suggested increasing the slope of the coincidence timing window allowance and reconfiguring the \texttt{phasetd} phase-time-amplitude histograms with specific tuning for early warning. This would allow coincidences to be made when single-detector trigger pairs fall outside the light-travel time. These adjustments could lead to more gravitational wave signals being detected in early warning, and we hope to demonstrate their effectiveness with future binary neutron star events.

We discussed the SNR optimiser used in the PyCBC Live search, as well as improvements to the SNR optimiser algorithm and hyperparameter optimisation performed during this PhD, highlighting the potential for future development in this area. Finally, I discussed my industrial placements during the course of the PhD, where I gained a wealth of knowledge and skills that I brought to my research.

The current epoch of gravitational wave astronomy is ripe for further improvements in search techniques, independent of detector upgrades. New methods can increase gravitational wave search sensitivity in parallel with detector advancements, and these techniques will be applicable to the next generation of detectors, which will explore an even larger gravitational wave parameter space. By enhancing our searches, we can observe more events, and in the case of early warning, provide as much notice as possible, thereby deepening our understanding of the Universe.
