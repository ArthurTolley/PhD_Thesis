As a DISCnet-funded PhD student, I was required to undertake two 3-month placements in industry. These placements were instrumental to my success as a PhD student: the skills, knowledge, and change in environment drove me to learn more in those six months than in some years of my PhD. The placements also helped inform my career decisions and clarified my post-PhD direction.

In this chapter, I will briefly describe the details of my employment and the projects I undertook during these placements. Due to non-disclosure agreements, confidentiality, and the passing of time, I am unable to provide extensive details or specific figures from my work. Section~\ref{8:sec:shell} covers my time at Shell Research UK, while Section~\ref{8:sec:MCA} discusses my work at the Maritime and Coastguard Agency, a UK Government civil service agency.

\section{\label{8:sec:shell}Shell Research UK}

I was hired as a Data Scientist Intern at Shell and placed within the Predictive Maintenance team. The team’s primary focus is to use advanced data science techniques to predict maintenance needs for the many machines at Shell sites globally, helping to prevent unplanned disruptions caused by failing components.

Predictive maintenance aims to anticipate equipment failures by analyzing operational data. A machine may refer to anything from a simple pump to a complex refinery system. Components in these machines are fitted with sensors that monitor parameters like temperature, pressure, and vibration. These readings form a high-dimensional dataset reflecting the machine's operational health. For example, a long-term rise in temperature may indicate wear or degradation, signaling an impending failure.

We employed a reconstruction error approach to identify failure patterns. This method involves training machine learning models to understand the normal behavior of a machine. Reconstruction error measures how closely the model's predictions match actual sensor readings. Significant deviations between predicted and observed values---i.e., a high reconstruction error---can indicate abnormal behavior and possible component failure.

The machine learning algorithms used in this approach fall under unsupervised learning, which does not rely on labeled data for classification. During my time at Shell, I developed and implemented Principal Component Analysis (PCA) as part of the predictive maintenance pipeline for monitoring machine components.

\section{\label{8:sec:MCA}Maritime and Coastguard Agency}

I worked as a Data Scientist Intern at the Maritime and Coastguard Agency (MCA), which is responsible for maritime safety, environmental protection, and coordinating search and rescue operations under the UK Department for Transport.

I was tasked with on-boarding a new Automatic Identification System (AIS) dataset, containing billions of data points tracking every vessel in UK coastal waters over the past four years. This dataset contained numerous errors and anomalies that needed to be processed efficiently and in parallel to ensure the data was usable by other teams within the agency.

At the MCA, I extensively used PySpark---a Python API for Apache Spark that supports large-scale data processing in distributed environments---and Databricks, a cloud platform that provides these environments. With these tools, I developed scalable pipelines to process billions of geospatial data points, impute missing values, calculate new information, and identify errors in the dataset. The unique nature of working with geospatial data allowed me to focus on identifying vessels moving anomalously over land, analyzing historical traffic trends in protected areas, and preventing vessel collisions at sea.
