% The searches



The sensitivity~\cite{aLIGO_design_curve:2018} of Gravitational Wave detectors is limited by fundamental sources of noise such as seismic noise~\cite{Glanzer:2023}, thermal noise~\cite{thermal_noise:2018} and quantum noise~\cite{quantum_noise:2003}. These are the stationary source of noise~\cite{PSD_var:2020}, we also have non-Gaussian noise~\cite{Noise_Guide:2020} which manifests as transient noise bursts in the data which can contribute to false positives in gravitational wave searches and the obscuring of real gravitational wave signals~\cite{GW170817:2017, GW150914_noise:2016}. There are numerous detector characterisation groups~\cite{O2O3_DetChar:2021, VirgoDetChar:2023} within the international gravitational wave network which work towards identification, characterisation and reduction of the noise transients~\cite{ArchEnemy:2023, Glanzer:2023, gravityspy:2017, gravityspy:2021, gravityspy:2023, glitschen:2021, Nuttall:2018, reducing_scattering:2020, BayesWave:2015, gwadaptive:2022, O3_subtraction:2022, Powell:2016} which negatively affect the data, improving the quality of the detector and enabling more robust searches for gravitational waves. Alongside making these improvements, the study of these noise sources can lead to detector improvements to enhance sensitivity and increase the rate at which we detect gravitational waves. Throughout my research I have contributed to the Detector Characterisation group within the LIGO Scientific Collaboration.

\section{Detector Characterisation Overview}

The broad remit of the detector characterisation group is the monitoring of the detector status and ensuring that the LIGO detectors are producing good quality data. The LIGO detectors have thousands of auxiliary channels alongside the main strain channel~\cite{iDQ:2020} which monitor the many subsystems of the detector and can be used to find noise correlations between the different parts of the instrument. These subsystems and subsequent correlations can be monitors with a number of detector characterisation tools, a few of which will be discussed later in this chapter.

The detector characterisation group has the important task of verifying data quality when a gravitational wave signal is detected in live time, giving the confirmation that the gravitational wave signal isn't a false alarm potentially caused by noise transients or poor data quality immediately surrounding the gravitational wave signal. To do this data quality check the detector characterisation group maintains a number of LIGO Summary Pages which display a daily summary of detector status information from the subsystems, environmental data and outputs of data quality tools, for each detector.

\section{Noise Transients}

What are they?

Noise transients, commonly referred to as glitches, are short duration bursts of noise found in the LIGO data which commonly occur and lower the quality of the data when they do~\cite{LIGO_data_quality:2015}. Glitches can be caused by environmental or instrumental factors and due to the high levels of detector sensitivity a very small change in configuration can have a large impact on the operating conditions of the instrument. For example, it is common for high winds at the site to physically shake the detector housing and subsequently hardware, producing low frequency noise which can be seen in the output channels. 

As previously mentioned, glitches can also mimic gravitational wave signals producing false alarms in our gravitational wave search pipelines~\cite{GWMimicking:2010}. This reduces the confidence in our results and increases the false alarm rates of all signals, meaning we are less confident and our gravitational wave catalogues contain fewer signals. It is therefore of extreme importance for the detector characterisation group to characterise new glitches, identify their sources and eliminate them before the contaminate the signal population too much.

We have a variety of glitches which lie in all parts of the frequency space in our data, some can be grouped into similar morphology glitches but others are unique. The most common glitches are: blips~\cite{blips:2019}, scattered light~\cite{ArchEnemy:2023} and whistles~\cite{glitschen:2021}, omegascans of these glitches can be seen in figure~ADD. There are however, many many more glitch types, which can be seen here:~\cite{gravityspy:2017}. The sources of these glitches have been studied, for example, whistles are caused by the beating of radio frequencies in the detector, however, some glitches are from unknown sources such as blips. Scattered light has been found from many different sources in the detector and great effort has been made to reduce the prescence of scattered light in the data~\cite{reducing_scattering:2020} but as of the fourth observing run some scattered light still remains.
%
% figure for glitches
%
\section{Detector Characterisation Tools}

\paragraph{Omega-scan}

is a whitened spectrogram with a high time and frequency resolution, created by the \verb|gwdetchar-omega|. These are the most common way to visualise the time series data in the time-frequency plane which can reveal key time-frequency relationships that are virtually impossible for a human to see in the time series. For a given event time \verb|gwdetchat-omega| is used by the detector characterisation group to visually identify any glitches in the data. An example of an omega-scan can be seen in figure~ADD.
%
%figure of an omega-scan

\paragraph{Omicron}

is used to detect short bursts of power in the main and auxiliary data streams~\cite{iDQ:2020}. The events generated by Omicron are assigned parameters such as event time, SNR, peak frequency, duration, bandwidth alongside others. These parameters can be used to classify the type of glitch which caused the trigger. If the peak frequency is in the low frequency region then it is more likely to be a particular glitch type.

\paragraph{GravitySpy}

is a citizen science machine learning tool for classifying glitches found in gravitational wave data~\cite{gravityspy:2017}. GravitySpy has been trained by volunteers on the website~ADD and the project itself can be found at~ADD.

GravitySpy takes millions of omega-scans of gravitational wave data which have been found to contain bursts of power. These omega-scans are images which potentially contain glitches, they are published on the GravitySpy project and volunteers are given these images and the option for which glitch they think is contained within the image. Once millions of these images have been classified the machine learning algorithm is trained and it can go further to classify the images without the need of volunteers~\cite{gravityspy:2021}.

GravitySpy was run on data from the second and third observing runs, where it was able to classify 23 different glitch categories. The downsides of GravitySpy are the constantly changing noise background of the gravitational wave detectors and the new glitch types emerging which need to be trained on with volunteering again~\cite{gravityspy:2023}.

Further extensions have been made to GravitySpy with new machine learning architecture which are capable of retraining on new glitch types without needing the volunteering. They are also able to do this and that and everything else: GWSpyTreeThing~\cite{GSpyNetTree:2023}


\paragraph{Data Quality Vetoes}

are used by the gravitational wave searches and are a flags applied to periods of data which contain potential data quality issues~\cite{DQ_vetoes:2017}. These periods of data are identified using tools such as \verb|Omicron| and when applied to the searches there is a significant reduction in the number of high SNR triggers that would've been found during these times.

The data quality vetoes are constructed using auxiliary channel information which has been found to be strongly correlated to instrumental noise. An example is a significantly elevated transient noise rate in the strain channel five days prior to GW150914~\cite{GW150914:2016} which was traced back to the 45MHz electro-optic modulator driver system used to generate optical cavity control feedback signals~\cite{aLIGO:2015}. The noise caused by this channel was given a category 1 veto data quality flag and removed 2.62\% of the total coincident time from the analysis period.

\paragraph{iDQ}

is another tools used to generate data quality flags which are used in the gravitational wave searches~\cite{iDQ:2020}. iDQ is a supervised learning framework which autonomously detects glitches based only on the auxiliary channels which are insensitive to gravitational waves---meaning these channels will only see glitches and not gravitational wave radiation. iDQ operates in low-latency and is used in the live gravitational wave searches to provide data quality information in real-time. This tool is invaluable for low latency searches to prevent significant contamination from high power bursts of noise that would contaminate the searches~\cite{LIGO_data_quality:2015}.

\section{Conclusion}

I have described the function of Detector Characterisation groups at the gravitational wave detectors and the tools they used to perform their tasks in ensuring the rapid and proper identification of gravitational wave signals in offline and in real time. 
