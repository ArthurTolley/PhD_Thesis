% General Relativity and Gravitational Waves
In the 20th century, Albert Einstein introduced the theory of \GR~\cite{Einstein_1:1914, Einstein_2:1914, Einstein_3:1914, Einstein_4:1915, Einstein_5:1916, Einstein_6:1917, Einstein_7:1936}, which describes gravity as the curvature of spacetime caused by massive objects, rather than as a force. In this framework, objects move along geodesics in curved spacetime. A key prediction of the theory is the existence of \gws~\cite{Einstein_8:1937, Einstein_9:1938, Einstein_10:1939, Einstein_11:1948}, which are perturbations in spacetime generated by the acceleration and collisions of massive bodies, such as black holes or neutron stars. These waves propagate at the speed of light and induce small changes in the proper distance between freely falling objects. The detector strain caused by \gws is extremely small, typically on the order of $10^{-22}$ for the most energetic sources, making direct detection highly challenging. Despite this, the observation of \gws has become a crucial tool for studying astrophysical phenomena and testing the predictions of \GR.

% First Direct Detection
The first direct detection of \gws occurred in 2015 with the observation of GW150914~\cite{GW150914:2016}, a \gwadj signal produced by the merger of two black holes. This event was detected by the Laser Interferometer Gravitational-Wave Observatory (LIGO) and the Virgo~\cite{aVirgo:2015} \gwadj observatory using ground-based interferometers. The observed signal matched the predictions of \GR~\cite{GW150914_TGR:2016} for the inspiral, merger, and ringdown phases of a binary black hole coalescence. This detection marked the beginning of \gwadj astronomy, providing the first direct evidence of black hole mergers and opening a new observational window to study the Universe.

% Expansion of Observations
Following the first detection of \gws, the global \gwadj detector network expanded with the inclusion of the KAGRA detector~\cite{KAGRA:2021}. The current International Gravitational-Wave Observatory Network (IGWN) collaboration has enabled more comprehensive and sensitive observations of \gwadj events. With a running total of 221 \gwadj events~\cite{gwtc1:2019, gwtc2:2021, gwtc21:2024, gwtc3:2023, 1OGC:2018, 2OGC:2020, 3OGC:2021, 4OGC:2021, Princeton_1:2019, Princeton_2:2019, Princeton_3a:2022, Princeton_3b:2023, gracedb_superevents:2024} from \cbcs, including two mergers of binary neutron star systems~\cite{GW170817:2017, GW190425:2020} and two neutron star–binary black hole mergers~\cite{nsbh:2021}, across four observing runs as of the drafting of this thesis.

% Thesis Scope
The work performed in this thesis focuses on improvements to the current search methods for \gws from data obtained by ground-based \gwadj observatories. Chapter~\ref{chapter:1-gravitational-waves} provides a simple derivation of \gws, a brief discussion of how the current generation of detectors operates, and the expected \gwadj emissions from a compact binary coalescence. This discussion is intended to give the reader a basic understanding of the background required to appreciate the motivation for the research conducted in later chapters; for texts with more detailed discussions, please refer to~\cite{Moore_book:2012, Maggiore_book:2007, Schutz_book:2009}. Chapter~\ref{chapter:2-searches} reviews the detection methods used for \gws, focusing on the PyCBC~\cite{PyCBC:2016} search pipeline. Chapter~\ref{chapter:3-detchar} concentrates on Detector Characterisation and the techniques employed to characterise and calibrate ground-based interferometer noise, producing high-quality data for further study.

Chapter~\ref{chapter:4-archenemy} investigates the feasibility of modelling \gwadj detector data artefacts, specifically the \scladj artefact, and subtracting these artefacts to clean the data before the search for \gws. In Chapter~\ref{chapter:5-pycbc-live}, we describe the improvements to the PyCBC Live search's ranking statistic aimed at enhancing detection sensitivity. Chapter~\ref{chapter:6-earlywarning} details the PyCBC Live Early Warning search, examining its deficiencies and suggesting improvements to ensure the identification of all potentially electromagnetically bright events. In Chapter~\ref{chapter:7-snr-optimiser}, we discuss the investigation and enhancements made to the PyCBC Live rapid detection statistic optimiser. Finally, in Chapter~\ref{chapter:8-industry}, we briefly discuss the industrial placements undertaken as a requirement of this doctoral programme.
