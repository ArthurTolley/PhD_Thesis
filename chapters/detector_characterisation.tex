The detector noise is approximated as Gaussian and stationary \cite{gaussian_noise}, this isn't true.
Data artefacts (commonly referred to as glitches) are non-Gaussian noise which inhibit the sensitivity of the
detector, reduce the quality of the data and, obscure \gw detections \cite{transient_noise}.

There are a number of common glitches in the detector data: blips \cite{blips}, whistles \cite{whistles},
scattered light \cite{Accadia}. The sources of these glitches are studied, for example, whistles are caused by the
beating of radiofrequencies in the detector, however, some glitches are from unknown sources (e.g. blips).

We can reduce the noise from the sources by making improvements to the detectors \cite{soni_scattered}, this lowers the
frequency of some glitches but doesn't remove them entirely. We can eliminate some glitches by completely removing the
source \cite{laura_char}. We are also able to reduce noise retroactively by gating certain times or using algorithms to
remove noise \cite{cit_science_soni}. An example of this is detailed in the section \ref{sec:scattered_light}.