This thesis presents advancements in the detection of \gws from compact binary coalescences, utilising the most sensitive observatories constructed to date. The research focuses on enhancing \gwadj signal searches through the development of new tools and the application of existing methodologies to increase the sensitivity of live \gwadj searches. 

We introduced a novel noise artefact model, which enabled the identification and removal of glitches, thereby facilitating the recovery of previously missed \gwadj injections. This pioneering approach established a glitch search pipeline that adapted techniques typically used in \gwadj searches to address the unique characteristics of glitches. Additionally, we implemented an exponential noise model within the PyCBC Live search framework, significantly improving the detection ranking statistics for \gwadj signals and demonstrating the potential for substantial increases in detection sensitivity.

Furthermore, we analysed and proposed enhancements for the PyCBC Live Early Warning search to maximise the detection of \gwadj events in the early warning regime. Our findings highlighted deficiencies in the current ranking statistic and led to recommendations for optimising coincidence timing windows and refining phase-time-amplitude histograms. These adjustments aim to increase the detection of \gwadj signals, particularly binary neutron star events, in early warning scenarios.

The results underscore the importance of advancing search techniques in \gwadj astronomy, which can operate independently of detector improvements. By refining search methodologies, we enhance the capacity to detect a greater number of events, contributing significantly to our understanding of the Universe.
