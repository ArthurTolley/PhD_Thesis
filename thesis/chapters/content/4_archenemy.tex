\section{\label{4:sec:ArchEnemy-intro}Introduction}

The Laser Interferometer Gravitational-Wave Observatory (LIGO)~\cite{aLIGO:2015} and Virgo~\cite{aVirgo:2015} collaborations made the first observation of \gws in September 2015~\cite{GW150914:2016}. The detection established the field of \gwadj astronomy and a global network of \gwadj detectors, now joined by KAGRA~\cite{KAGRA:2021}, has allowed for the detection of approximately 100 \gwadj events~\cite{gwtc1:2019, gwtc2:2021, gwtc21:2024, gwtc3:2023, 1OGC:2018, 2OGC:2020, 3OGC:2021}.

The detection of \gws is made possible by both the sensitivity of the detectors and the search pipelines~\cite{PyCBC:2016, GstLAL:2020, SPIIR:2020, MBTA:2021, cWB:2020} which analyse raw strain data from the output of the detectors and identify observed \gwadj signals. One of the problems that these search pipelines must deal with is the fact the data contains both non-stationary noise and short duration `glitches'~\cite{Noise_Guide:2020, O2O3_DetChar:2021, VirgoDetChar:2023} where noise power increases rapidly. Glitches are caused by instrument behaviour or interactions between the instrument and the environment~\cite{GW150914_noise:2016, Glanzer:2023} and glitches reduce the sensitivity of the detectors~\cite{O3_sensitivity:2020}, can potentially obscure candidate \gwadj events~\cite{gwtc2:2021} and can even mimic \gwadj events~\cite{GWMimicking:2010, PyCBC_singles:2022}.

Different classes of glitches have been characterised using tools such as Gravity Spy~\cite{gravityspy:2017, gravityspy:2023}. Of the 325,101 glitches classified by Gravity Spy in the third observing run of Advanced LIGO~\cite{gravityspy:2021} with a confidence of $90\%$ or higher, 120,733 ($32.1\%$) were classified as ``Scattered Light''. \Scladj glitches occur in the 10-120Hz frequency band~\cite{reducing_scattering:2020} which coincides with the frequency band where we observe the inspiral and merger signatures of compact binary coalescences (CBCs). \Scladj glitches are characterised by an arch-like pattern in a time-frequency spectrogram of the detector output, as seen in Figure~\ref{4:fig:scattered_light}. 
%
\begin{figure}
  \includegraphics[width=\textwidth]{images/4_archenemy/Section1/single_stack.pdf}
  \caption{An Omega scan \cite{gwdetchar_tools:2021} of \gwadj data containing an example of a \scladj glitch. \Scladj glitches are characterised by a symmetric arch-like pattern. Multiple \scladj glitches can be seen at the 4, 8 \& 12 second marks as well as multiple harmonic glitches at 8 seconds.}
  \label{4:fig:scattered_light}
\end{figure}
%
\Scladj glitches occur when laser light in the interferometer is scattered from the main optical path by components within the detector. The motion of these components is coupled to seismic motion, inducing a phase shift on the light being scattered as the surface moves back and forth. This \scl then recombines with the main laser, producing \scladj glitches in the data. The surfaces from which \scladj glitches originate have been objects on optical benches such as lenses, mirrors and photo-detectors~\cite{TAccadia:2010}.

\Scladj glitches have been a significant problem when observing CBCs. As an example, GW190701\_203306 was coincident with a \scladj glitch, as shown in figure \ref{4:fig:obscured_detection}~\cite{gwtc2:2021}, requiring subtraction from the data before the event could be properly characterised~\cite{O3_subtraction:2022}. A further 7 candidate events were found to be in coincidence with \scladj glitches in the third observing run~\cite{gwtc3:2023}. For this reason, it is important to reduce the effect of \scladj glitches in the detectors and \gwadj search pipelines. \Scladj glitches occur as single or multiple glitches and can appear rapidly in time and simultaneously in frequency (see figure \ref{4:fig:consec_scattered_light}), which we refer to as harmonic glitches.

\begin{figure}
  \includegraphics[width=\textwidth]{images/4_archenemy/Section1/GW190701_203306_overlay.pdf}
  \caption{GW190701\_203306, a \gwadj event coincident with a \scladj glitch in the data from the LIGO Livingston observatory. The orange dashed track shows the inferred time-frequency evolution of a \gwadj event produced by a compact binary coalescence, the red solid line is an overlaid track of the approximate location of the coincident fast scattering glitches.}
  \label{4:fig:obscured_detection}
\end{figure}

\begin{figure}
  \includegraphics[width=\textwidth]{images/4_archenemy/Section1/multiple_harmonics.pdf}
  \caption{An Omega scan \cite{gwdetchar_tools:2021} of \gwadj data containing multiple examples of \scladj glitches. Here we see multiple \scladj glitches repeating periodically in a $20$ second period of time. Harmonic \scladj glitches are also seen in multiple stacks, the harmonic glitches share \emph{time period} values and the \emph{glitch frequency} values are $n-$multiples of the lowest frequency harmonic within the stack.}
  \label{4:fig:consec_scattered_light}
\end{figure}

The most obvious way to remove the impact of glitches is removing the mechanisms which produce the glitches in the observatories. This has been investigated in previous works~\cite{reducing_scattering:2020, TAccadia:2010, Nuttall:2018, gwadaptive:2022, HilbertHuang:2017, tvf-EMD:2020, Scattering_Monitoring:2022, Was_Subtract:2021}, which focus on identifying the surfaces in which light is being scattered from and then mitigating the scattering by reducing the reflectivity of the surface, seismically isolating it or relocating it. 

An alternative method for reducing \scladj glitches, known as ``RC tracking'', was implemented in the Advanced LIGO observatories in January 2020~\cite{reducing_scattering:2020}. The Advanced LIGO detectors employ a quadruple pendulum suspension for the test masses, where two chains suspend four masses in this suspension system, one for the test mass optic and the other for the reaction mass. The reaction mass is used to impose a force upon the test mass, and a significant source of \scladj glitches was the large relative motion between the test mass chain and the reaction chain. To mitigate this effect, the relative motion between the end test mass and the reaction mass needed to be reduced. This was achieved by ensuring the two chains are moving together by applying force to the top stage of the quadruple suspension system in Advanced LIGO. The implementation of RC tracking represented a decrease from $0.01 s^{-1}$ to $0.0001 s^{-1}$ and $0.0072 s^{-1}$ to $0.0012 s^{-1}$ in the number of \scladj glitches detected by Gravity Spy for LIGO-Hanford and LIGO-Livingston, respectively.

While methods for preventing \scladj glitches have been developed and have shown success, they have not been able to remove the problem of \scladj glitches from the data. Additionally, as the detectors continue to be upgraded and increase in sensitivity, new sources of \scladj glitches will continue to appear. Identifying these new sources and mitigating their effects can take many months, during which time the detectors are taking in data which might be affected by the presence of \scladj glitches. Therefore, it is not realistic to believe that analyses will be able to regularly run on data that does not contain \scladj glitches and this motivates us to develop a technique for mitigating the impact of these glitches when trying to identify CBCs in \gwadj data.

In this work, we present a new method for identifying and removing \scladj glitches from \gwadj data in advance of running searches to identify CBCs.
We first introduce a method for identifying when \scladj glitches are present in detector data, through the creation of a new modelled search for \scladj glitches, similar to how we search for \gws using matched filtering. We can model \scladj glitches, generate a suitable set of glitch waveforms and perform a matched filter search on detector data. We then subtract identified glitches from the data to increase detector sensitivity. The detector data isn't Gaussian and stationary, so the matched filter does have the potential to identify non-\scladj glitches, and potentially even \gwadj signals, as \scladj glitches. To prevent this we also demonstrate a new \scl $\chi^{2}$ test, which can distinguish between \scladj glitches, and other glitches---and \gwadj signals---in the data.

We begin by reviewing previous research and describing the formulation of the waveform model used for characterizing \scladj glitches in section \ref{4:sec:sc_li}. In section \ref{4:sec:search_techniques} we introduce the various techniques used in the search to identify \scladj glitches in \gwadj data and the results of the \scladj glitch search. In section \ref{4:sec:results} we describe the results of a ``glitch-subtracted'' \gwadj search and any increases in sensitivity. We conclude in section \ref{4:sec:conclusion} and discuss the implementation of this method in future observing runs.

\section{\label{4:sec:sc_li}\Scl}

To identify \scladj glitches in \gwadj data requires an accurate model of \scladj glitches. This section details the derivation of the model we will use for generating our \scladj glitch filter waveforms, along with its parameterisation.

\subsection{Modelling \scladj glitches — a review}

Our model for \scladj glitches draws heavily from~\cite{TAccadia:2010}, and we briefly review the main details of the model presented there. In~\cite{TAccadia:2010}, the authors construct a model to accurately predict the increase in noise due to \scl during periods of increased micro-seismic activity. The model in~\cite{TAccadia:2010} is constructed from parameters related to physically measurable properties of the detector such as the mirror transmission factor, $T$, the finesse of the Fabry–Pérot cavity, $F$, or the wavelength of the light, $\lambda$.

They define the amplitude of the additional beam produced by light scattering off of a surface as
%
\begin{equation}
    A_{sc} = A_{0} T \sqrt{\frac{2 F}{\pi}} \sqrt{f_{sc}} e^{i \phi_{sc}}\,,
    \label{4:eqn:accadia_amplitude}
\end{equation}

%
where $A_{0}$ is the amplitude of the light resonating in a Fabry–Pérot cavity, $f_{sc}$ is the fraction of the optical power scattered back into the main beam and $\phi_{sc}$ is the phase angle modulated by the displacement of the scattering optics, defined as
%
\begin{equation}
    \phi_{sc}(t) = \frac{4 \pi}{\lambda} ( x_{0} + \delta x_{opt}(t) ),
    \label{4:eqn:accadia_phase_noise}
\end{equation}
%
where $\delta x_{opt}$ is the displacement of the scattering surface and $x_0$ is the static optical path.

The total amplitude of the beam inside the arm is given by $A_{tot} = A_{0} + A_{sc}$, with a phase angle equal to the phase noise introduced by the \scl $\delta \Phi = \frac{A_{sc}}{A_{0}} \cdot \sin \phi_{sc}$. The noise introduced by the \scl, $h_{sc}$, can be approximated through the relationship $\sin(\phi+ \delta\phi) \approx \sin\phi + \cos\phi \cdot \delta\phi$ for small $\delta\phi$, and can be expressed as
%
\begin{equation}
    h_{sc}(t) = G \cdot \sin \left(\frac{4 \pi}{\lambda} (x_{0} + \delta x_{sc}(t) ) \right),
    \label{4:eqn:accadia_strain}
\end{equation}
%
where $G$ is the \textit{coupling factor}, defined as $K \cdot \sqrt{f_{sc}}$ where
%
\begin{equation}
K = \frac{\lambda}{4 \pi} \frac{T}{\sqrt{2 F \pi}}.
\end{equation}
%

The displacement of the scatterer when presenting with oscillatory motion is then given as
%
\begin{equation}
    \delta x_{sc} (t) \simeq A_{m} \sin(2 \pi f_{m} t),
    \label{4:eqn:accadia_oscillatory}
\end{equation}
%
where $f_{m}$ is the frequency-modulated signal with modulation index $m = A_{m} \frac{4 \pi}{\lambda}$ and where $A_{m}$ is the amplitude of the $n$th harmonic. Finally, Equation~\ref{4:eqn:accadia_strain} can be simplified when considering only small bench motion, according to
%
\begin{equation}
    h_{sc}(t) = G \cdot \cos\phi_{0} \cdot \frac{4 \pi}{\lambda} \cdot \delta x_{sc}(t) + h_{\text{offset}}.
    \label{4:eqn:accadia_strain_linearised}
\end{equation}

\subsection{Model}

The model introduced in~\cite{TAccadia:2010} for the \gwadj strain noise introduced by \scl uses a lot of knowledge about the detector state. The model used in this work will be more phenomenological in the parameterisation, allowing us to rely only on the characteristics of the glitches in the strain data and not knowledge of the detector configuration, especially in cases where this detector information might not be known. Each \scladj glitch, as viewed in a spectrogram (see figure \ref{4:fig:scattered_light}), has two easily identifiable features: the maximum frequency reached, \emph{glitch frequency} ($f_{gl}$); and the period of time the glitch exists in detector data, \emph{time period} ($T$).  In addition, we can fully describe an artifact by defining an \emph{amplitude} ($A$), \emph{phase} ($\psi$) and \emph{centre time} of the glitch ($t_0$).

To formulate a model of \scladj glitches in terms of these parameters, we simplify equation \ref{4:eqn:accadia_strain}, treating the strain noise caused by \scl as the sinusoidal function
%
\begin{equation}
  h_{sc}(t) = A \sin\left(\frac{4\pi}{\lambda}x(t) + \psi\right),
  \label{4:eqn:h_sc_initial}
\end{equation}
%
where $A$ is some constant, $x(t)$ is the displacement of the scattering surface and $\psi$ is an offset.

\Scladj glitches are caused by the physical increase in the distance travelled by the light as a consequence of being reflected off of a surface. We assume a simple harmonic oscillator for the surface displacement with frequency $f_{rep}$
%
\begin{equation}
    x(t) = B \sin\left(2\pi f_{rep}t\right),
    \label{4:eqn:harmonic_displacement}
\end{equation}
%
which is directly related to the \emph{time period}, $T$,
%
\begin{equation}
  f_{rep} = \frac{1}{2 T}.
  \label{4:eqn:f_rep}
\end{equation}
%
The \emph{time period} of a \scladj glitch only corresponds to half of a sinusoidal wave hence the multiplier of 2 on the denominator of equation \ref{4:eqn:f_rep}. We calculate the instantaneous frequency of the strain noise~\cite{TAccadia:2010}
%
\begin{equation}
    f(t) = \left|\frac{1}{2\pi} \frac{d}{dt}\left(\frac{4\pi}{\lambda}x(t) + \psi\right)\right|,
    \label{4:eqn:instant_freq}
\end{equation}
%
which we can use to predict the frequency of each scattered light glitch. We define the maximum frequency of each glitch (at the peak of the arch) as the glitch frequency, $f_{gl}$, which can be found by taking the maximum of Equation~\ref{4:eqn:instant_freq}
%
\begin{equation}
    f_{gl} = \text{max}\left[f(t)\right],
\end{equation}
%
giving an equation for the glitch frequency of a scattered light glitch
%
\begin{equation}
    f_{gl} = B \frac{4\pi}{\lambda} f_{rep}.
    \label{4:eqn:glitch_frequency}
\end{equation}
%
Rearranging Equation~\ref{4:eqn:glitch_frequency} yields the expression $\frac{4\pi}{\lambda} = \frac{f_{gl}}{f_{rep} B}$ and when combined with Equations~\ref{4:eqn:h_sc_initial} and~\ref{4:eqn:harmonic_displacement} gives the final form of the strain noise induced by scattered light
%
\begin{equation}
  h_{sc}(t) = A \sin\left(\frac{f_{gl}}{f_{rep}} \sin(2 \pi f_{rep} t) + \psi\right),
  \label{4:eqn:model}
\end{equation}
%
where $A$ and $\psi$ are amplitude and phase parameters to be maximised over.

In~\cite{Was_Subtract:2021} they provide another term to describe \scladj glitches which uses the instantaneous frequency as a function of time, allowing the identification of the correct amplitude at all frequencies. This term is due to radiation pressure coupling and is thought to be dominant at low frequencies. The relationship between our amplitude and the new amplitude depends on the power in the arm cavities and the signal recycling mirror reflectivity, which we do not consider in our model and so disregard this extra term.

\subsection{Harmonics}

As described in~\cite{TAccadia:2010}, harmonic glitches appear at the same time with different \emph{glitch frequencies}. A harmonic glitch is a glitch with a \emph{glitch frequency} that is a positive integer multiple of the glitch frequency of the glitch in the stack with the lowest glitch frequency value. This lowest frequency glitch has the potential to appear below $15$Hz and will be masked by other sources of noise and therefore cannot be seen. An example of harmonics can be seen in figure \ref{4:fig:consec_scattered_light}.

%Section
\section{\label{4:sec:search_techniques}Identifying \scladj glitches in \gwadj strain data}

Equipped with our model for \scladj glitches from the previous section, we now discuss how we identify and parameterize \scladj glitches in a stretch of \gwadj data before we apply this to searches for CBCs in the next section. 

\subsection{\label{4:subsec:MF}Matched filtering}

Given our model of \scladj glitches, we can consider a realisation with specific values of the parameters discussed above. To identify glitches with these values of the parameterisation we use matched filtering, described in Section~\ref{2:sec:matched-filter}. \Scladj glitches will take a range of values of the parameters describing them and we must be able to identify glitches anywhere in the parameter space. Following~\cite{FINDCHIRP:2012} we can analytically maximize over the \emph{amplitude}, \emph{phase} and the \emph{centre time} glitch parameters in equation \ref{4:eqn:model} exactly as described in Section~\ref{2:sec:matched-filter}.

\subsection{Template bank}

Our \scladj glitch model is parameterised by 5 variables. As shown above, by maximizing over \emph{phase}, \emph{amplitude} and \emph{time}, we can analytically maximize the SNR over 3 of these variables. However, the remaining two describe the intrinsic evolution of the glitch, and we must explicitly search over these parameters. To do this we create and use a ``template bank'' of \scladj glitch waveforms, created such that it would be able to identify glitches with any value of our 2 remaining parameters, \emph{glitch frequency} and \emph{time period}.

To generate this template bank, a stochastic template placement algorithm~\cite{Harry_sbank:2009} is used. This algorithm randomly generates a new template, places the template in the existing template bank and calculates the ``distance'' (how similar two templates appear) between the new template and existing templates. If the new template is too close to an existing template it is discarded, otherwise, it is accepted into the bank. The density of the bank is dependent on the allowed distance between two templates. The dominant cost of the search for \scladj glitches is matched filtering and is approximately linear in the number of templates, a larger template bank will find all the glitches with more accurate parameter values, but the computational cost of the search will be increased. The template bank generation alone does not constitute a significant computational cost in the search.

The distance function we have chosen to evaluate templates by is the match between two glitch templates. To calculate the match, we first normalize both templates such that the matched filter between a template and itself would be equal to 1. We can then compute the inner product between the two normalised templates to find their match
%
\begin{equation}
    M = \max_{t, \psi} (\hat{a} | \hat{b}).
  \label{4:eqn:match}
\end{equation}
%
The value for the match, $M$,  is bounded between 0 and 1, where a value of 0 indicates orthogonal waveforms and a value of 1 indicates identical normalised waveforms. The maximum match allowed between any two templates in our bank is 0.97, which implies that a fully converged stochastic bank would have at least one waveform in it with $M \geq 0.97$ for any point in the space of parameters. 

For our \scladj glitch search, we generated a template bank with \emph{time periods} $\in$ $1.8 \, \text{s}\text{--}5.5 \, \text{s}$ and \emph{glitch frequencies} $\in$ $20 \, \text{Hz}\text{--} 80 \, \text{Hz}$. We chose to use the Advanced LIGO zero-detuned high-power sensitivity curve~\cite{aLIGO_design_curve:2018}\footnote{The zero-detuned high-power sensitivity curve is a broader noise curve than the O3 Advanced LIGO data that we identify \scladj glitches in. However, this broader noise curve will result in \textit{more} template waveforms than we need, and will therefore overcover the parameter space, rather than potentially miss \scladj glitches.}. This allowed us to generate a template bank that contains 117,481 templates. We visualize the distribution of the templates as a function of \emph{time period} and \emph{glitch frequency} in Figure~\ref{4:fig:sq_bank}, observing a greater density of templates at higher frequencies and longer durations, indicating that templates in this region look less similar and therefore, more templates are needed to fully describe the parameter space.

\begin{figure}
  \centering
  \includegraphics[width=1.0\textwidth]{images/4_archenemy/Section3/3.2/template_bank_square.pdf}
  \caption{The template bank of \scladj glitch templates (black points) used in the search for \scladj glitches. The \emph{glitch frequency} parameter values range from $20 \, \text{Hz}\text{--} 80 \, \text{Hz}$ and the \emph{time period} parameter values range from $1.8 \, \text{s}\text{--} 5.5 \, \text{s}$. This bank was created with a maximum match allowed between two templates of $0.97$ and contains 117,481 templates.}
  \label{4:fig:sq_bank}
\end{figure}

\subsection{Identifying potential \scladj glitches in the data}

We test our method by searching through \gwadj data from 2019-11-18 16:13:05--2019-11-25 22:11:09 for the LIGO-Hanford and LIGO-Livingston detectors. This time corresponds to the 25th period of data used in the LVK analysis of O3 data for CBCs~\cite{gwtc3:2023} and is prior to the implementation of RC tracking~\cite{reducing_scattering:2020} in these detectors. We only analyse data that is flagged as ``suitable for analysis" on the Gravitational Wave Open Science Center~\cite{GWOSC:2021}\footnote{We note that in O3 \emph{only} data suitable for analysis is released, so we simply analyse all the publicly available data.}. This corresponds to 5.96 days of analysable data for LIGO-Hanford and 5.93 days for LIGO-Livingston.

Equipped with our template bank, we now identify potential \scladj glitches in the data. We matched filter all the data against all the templates, producing a SNR time series for every template in the bank. These SNR time series will contain peaks which, when above a certain limit, indicate a potential \scladj glitch at a particular time. We retain any maxima within the time series that have a SNR greater than 8. However, as we do this independently for every template, we will identify multiple peaks for any given glitch, and we will also find peaks that correspond to other types of glitch, or even \gwadj signals. We discuss how we reduce this to a list of identified \scladj glitches in the following subsections.

\subsection{Scattered-light signal consistency test}

To prevent the search for \scladj glitches from misclassifying other classes of glitches, or \gwadj signals, we employ a $\chi^2$ consistency test. The $\chi^2$ discriminator introduced in \cite{Allen_Chi:2005} divides \gwadj templates into a number of independent frequency bins. These bins are constructed so as to contain an equal amount of the total SNR of the original matched filter between template and data. The $\chi^{2}$ value is obtained by calculating the SNR of each bin, subtracting this from the expected SNR value for each bin and squaring the output. These values are summed for all bins and this value forms the $\chi^{2}$ statistic,
%
\begin{equation}
  \chi_{r}^{2} = \frac{\chi^{2}}{\textrm{DOF}} = \frac{n}{2n - 2} \sum_{i=1}^n \left(\frac{\rho}{\sqrt{n}} - \rho_{bin,i}\right)^2.
  \label{4:eqn:chi_squared}
\end{equation}
%
Here $n$ is the number of bins, $\rho$ is the SNR of the original matched filter between template and data, and $\rho_{bin}$ is the value of the SNR found when matched filtering one of the divided templates and the data. This test is constructed so as to produce large values when the data contains a glitch, or astrophysical signal, that is not well described by the template, but to follow a $\chi^2$ distribution if a glitch that matches well to the template is present, or if the data is Gaussian and stationary.

The $\chi^2$ test that we employ is similar to that of \cite{Allen_Chi:2005}, however CBC waveforms increase in frequency over time whereas \scladj glitch templates are symmetric about their centre. We therefore choose to construct our $\chi^2$ test with four non-overlapping bins \emph{in the time domain}, each of which contributes equally to the SNR, an example of the split template can be seen in figure \ref{4:fig:split_temp_subplot}. The matched filter between the bins and data is computed, and the $\chi_{r}^{2}$ value is calculated using equation \ref{4:eqn:chi_squared} (where $n=4$). Any glitch, or astrophysical signal, which does not exhibit symmetric morphology should not fit with this deconstruction of the template, and should result in elevated $\chi_{r}^{2}$ values.

\begin{figure}
  \centering
  \begin{minipage}[t]{1.0\linewidth}
  \includegraphics[width=0.49\textwidth]{images/4_archenemy/Section3/3.4/split_bins.pdf}
  \hspace{0.01\linewidth}
  \includegraphics[width=0.49\linewidth]{images/4_archenemy/Section3/3.4/split_bins_qscan.pdf}
  \end{minipage}
  \caption{A \scladj glitch template (left) where the colours and line-styles are indicative of the four equal SNR time bins to be used in calculating the $\chi_{r}^{2}$ value and re-weighting the SNR. The same \scladj glitch template bins overlaid on an injection of the \scladj glitch (right). The inner two bins are considerably shorter than the outer two bins, which informs us that the centre---higher frequency---region of the \scladj glitch contributes a larger amount to the SNR per unit time than the lower frequency regions.}
  \label{4:fig:split_temp_subplot}
\end{figure}

After computing the $\chi_{r}^{2}$ value for potential \scladj glitches, we follow~\cite{rw_snr_eq:2012} to compute a ``re-weighted SNR'', which is an empirically tuned statistic depending on the SNR and the $\chi_{r}^{2}$ value.  The re-weighting function we use matches that presented in~\cite{rw_snr_eq:2012},
%
\begin{equation}
\rho_{rw} =  \left\{  \begin{array}{l@{\quad}cr} 
\rho & \mathrm{if} & \chi_{r}^{2} \leq 1, \\  
\rho [(1 + (\chi_{r}^{2})^3)/2]^{-\frac{1}{6}} &  \mathrm{if} & \chi_{r}^{2} \ge 1,   
\end{array}\right.
\label{4:eqn:reweighting}
\end{equation}
%
where $\rho_{rw}$ represents the re-weighted SNR of the \scladj glitch template calculated using the SNR, $\rho$, and the $\chi_{r}^{2}$ value of that template.
We do note that this re-weighting function has been tuned for CBC waveforms, and we did not repeat that tuning here with \scladj glitches. One could retune this statistic, specifically targeting \scladj glitches, using (for example) the automatic tuning procedure described in~\cite{McIsaac_Chi:2022}. However, we demonstrate the suitability of the $\chi^{2}$ test for our purposes in Figure~\ref{4:fig:chi_snr} where we show the $\chi_{r}^{2}$ vs SNR distribution of the triggers found by our \scladj glitch search when performed on data containing only \scladj glitches and data containing a binary black hole \gwadj injection.

\begin{figure}
  \makebox[\textwidth][c]{\includegraphics[width=\textwidth]{Thesis/images/4_archenemy/Section3/3.4/SLvsBBH_6.pdf}}
  \caption{The signal-to-noise ratio (SNR) and $\chi_{r}^{2}$ values for the triggers identified by the matched filtering and clustering of the \scladj glitch template bank with data containing only \scladj glitches (grey hexagons) and data containing a binary black hole \gwadj injection (red triangles). The black contour lines represent the re-weighted SNR values the trigger will take when equation \ref{4:eqn:reweighting} is applied. The orange dashed vertical line indicates the SNR value limit of 8, above which we decide to perform the $\chi^{2}$ test and calculate the re-weighted SNR. The blue solid contour line indicates a re-weighted SNR value of 8, which is the limit at which we consider the trigger to be real. Different re-weighting parameter values will produce different contour line shapes. It can be seen that no triggers for the data containing the \gwadj injection lie beneath the contour line, and therefore no \scladj glitches are found on the \gwadj signal.}
  \label{4:fig:chi_snr}
\end{figure}

We note that the $\chi^{2}$ test increases the number of matched filters required by the search and therefore the computational cost of the search. Each template would require the matched filtering of an additional $4$ ``binned'' templates to calculate the $\chi_{r}^{2}$ value to re-weight the SNR time series of that template, increasing computational cost by a maximum factor of $5$. However, we reduce this increase by only computing the $\chi^{2}$ where it is needed, specifically for any template where the SNR time series has values above the threshold of 8.

\subsection{Identifying all \scladj glitches in the data}

\begin{figure}
      \begin{minipage}[t]{1.0\linewidth}
        \includegraphics[width=0.49\linewidth]{images/4_archenemy/Section3/3.5/cluster_qscan.pdf}
        \hspace{0.01\linewidth}
        \includegraphics[width=0.49\linewidth]{images/4_archenemy/Section3/3.5/cluster_snr_ts.pdf}
      \end{minipage}
      \begin{minipage}[t]{1.0\linewidth}
        \includegraphics[width=0.49\linewidth]{images/4_archenemy/Section3/3.5/cluster_all_triggers.pdf}
        \hspace{0.01\linewidth}
        \includegraphics[width=0.49\linewidth]{images/4_archenemy/Section3/3.5/cluster_in_frequency.pdf}
      \end{minipage}
    \caption{The process for identifying all the \scladj glitches in a period of \gwadj data. The red overlay in the \gwadj data used in this example (top left) indicates the highest re-weighted signal-to-noise trigger found, the dashed vertical lines represent the time slice window around this trigger. The re-weighted SNR time series resulting from the matched filter of this trigger's template and the data (top right) is clustered in time to identify the triggers found above a signal-to-noise threshold of $8$, indicated by red vertical dashed lines. All the triggers from all the templates in the template bank are then clustered in time (bottom left) to identify the highest re-weighted signal-to-noise glitches in the data, indicated by the orange vertical dashes lines. The triggers found within the time slice window, with a similar \emph{time period} value---within $\pm 10\%$---of the highest re-weighted SNR trigger (bottom right) are clustered by their \emph{glitch frequency} value to find the harmonic glitches at the same time, indicated again by red dashed lines.}
    \label{4:fig:clustering_story}
\end{figure}

Our matched filtering process retains ``triggers'' (potential \scladj glitches) wherever the re-weighted signal-to-noise time series is larger than 8. We retain no more than one trigger within a window size equal to half the \emph{time period} of the template used to produce the re-weighted signal-to-noise time series, and only store triggers at local maxima. A re-weighted signal-to-noise time series with multiple peaks and identified triggers can be seen in the top right panel in figure \ref{4:fig:clustering_story}.

After matched filtering all the templates against the data, we will recover multiple triggers for any potential \scladj glitch, as we might expect to independently identify peaks in multiple templates around the true values of the glitch. We therefore collect all the triggers generated by the template bank and cluster these in time, using a window of half of the shortest duration template---$0.9$ seconds. This will result in a list of triggers corresponding to the highest re-weighted SNRs, where each trigger should correspond to a unique \scladj glitch. The bottom left panel in figure \ref{4:fig:clustering_story} shows an example of the triggers found by the search and the highest re-weighted SNR triggers found by clustering.

However, we also expect to see instances of harmonic glitches which are produced by the same scattering surface and so share the same \emph{time period} and have \emph{glitch frequency} values equal to a multiple of the lowest frequency glitch in the harmonic stack~\cite{TAccadia:2010}. We investigate each trigger in the list, searching for harmonic glitches occurring at the same time. We use the first list of triggers identified by all templates across the bank and filter by those that occur within $\pm0.05$ seconds of the \emph{centre time} of the trigger we are investigating, an example of this window can be seen in the top left panel of figure \ref{4:fig:clustering_story}. We then filter the triggers again, keeping only those with an associated \emph{time period} value within $\pm 10 \%$ of the trigger's \emph{time period}. Finally, we cluster these remaining triggers by their associated \emph{glitch frequency} using a window size of $4$Hz, a lower limit on the frequency separation of harmonic glitches, the bottom right panel in figure \ref{4:fig:clustering_story} shows the identification of harmonic glitches for the overlaid \scladj glitch in the top left panel of figure \ref{4:fig:clustering_story}.

\subsection{Hierarchical subtraction to find parameter values}

We now have a list of identified \scladj glitches and their parameter values, however, these might not be fully accurate when there are many glitches close in time and frequency, as illustrated in Figure~\ref{4:fig:overlay_goods}. It is important that the parameters we find match well with the glitches in the data to remove as much power as possible.

To better identify the parameter values of the \scladj glitches, we perform a hierarchical procedure using information about the glitches we have found so far. Firstly, we create new segments of time which we know contain \scladj glitches, taking a window of $8$ seconds on either side of each previously identified glitch, if two glitch windows overlap they are combined into the same segment.

For each segment, we then create a reduced template bank, consisting of templates ``close'' to the \scladj glitches previously identified in the segment. We take the smallest and largest \emph{time period} and \emph{glitch frequency} glitches in the segment and bound the retained templates by these values with $\pm 0.25$ seconds on the \emph{time period} and $\pm 1$Hz on the \emph{glitch frequency}. For each data segment the reduced template bank is matched filtered with the data, the maximum re-weighted SNR value is found, and the corresponding glitch is subtracted. We then matched filter \emph{again} and remove the next largest re-weighted SNR template. This process is repeated until no templates, when matched filtered with the data, produce any re-weighted SNR values above the SNR limit of $8$. This method of hierarchical subtraction produces our final list of \scladj glitches.

A further benefit of using these shorter data segments is that we are estimating the PSD of the data using only a short period of data close to the \scladj glitches being removed. This protects us from a rapidly changing PSD in non-stationary data, which might cause Gaussian noise to be identified with larger SNR in the periods where the PSD is larger. This can be resolved by including the variation in the power spectral density as an additional statistic in the re-ranking of triggers, which has been done for CBC \gwadj searches in~\cite{PSD_var:2020}.

We demonstrate the hierarchical subtraction step on an amount of data which contains four injected \scladj glitches in a single harmonic stack, this can be seen in figure \ref{4:fig:injected_glitches}. As shown, the \scladj glitches are found and subtracted from the data leaving behind a cleaned segment of \gwadj data with no excess noise. Figure~\ref{4:fig:overlay_goods} shows the identified \scladj glitch triggers before and after performing the hierarchical subtraction step on a stretch of data containing real \scladj glitches. We identify more triggers prior to performing hierarchical subtraction, however there are more errant mismatches between \scladj glitches and the overlaid templates. By performing the hierarchical subtraction, we more accurately identify \scladj glitches, however, we miss some glitches that were previously identified. There is still some imperfection in this process, and we do not refine the method further in this work, but highlight this as useful direction for future studies in removing \scladj glitches. 

\begin{figure}
     \centering
     \begin{minipage}[t]{1.0\linewidth}
        \includegraphics[width=0.49\linewidth]{images/4_archenemy/Section3/3.6/overlay_good_overlays_first.pdf}
        \hspace{0.02\linewidth}
        \includegraphics[width=0.49\linewidth]{images/4_archenemy/Section3/3.6/overlay_good_overlays_second.pdf}
     \end{minipage}
         \caption{LIGO-Hanford data from 2019-11-23 17:54:22--2019-11-23 17:55:12 containing \scladj glitches which have been identified by the ArchEnemy search (left), there is a misalignment in the template found for a number of glitches in this period of data and some missed glitches. Instead of finding all \scladj glitches at once we run a hierarchical subtraction to find the \scladj glitches in a chunk of time one-by-one. \Scladj glitches remaining after running the hierarchical subtraction search (right) for the same period of data, we have missed more \scladj glitches however misalignments have been removed. The highest harmonic at approximately $892$ seconds has been incorrectly split into two separate templates.}
    \label{4:fig:overlay_goods}
\end{figure}

\begin{figure}
     \centering
     \begin{minipage}[t]{1.0\linewidth}
        \includegraphics[width=0.49\linewidth]{images/4_archenemy/Section3/3.6/injected_artefacts_initial.pdf}
        \hspace{0.02\linewidth}
        \includegraphics[width=0.49\linewidth]{images/4_archenemy/Section3/3.6/injected_artefacts_subtracted.pdf}
     \end{minipage}
         \caption{Data containing an injected stack of harmonic \scladj glitches (left) and the corresponding data found when running the hierarchical subtraction search and subtracting the identified \scladj glitches from the data (right).}
    \label{4:fig:injected_glitches}
\end{figure}

\subsection{Identified \scladj glitches}

The methodology described in previous sections is implemented in our ``ArchEnemy'' pipeline, which is capable of searching for \scladj glitches in \gwadj data using a pre-generated bank of glitch templates. We use ArchEnemy to analyse the aforementioned data, which produced a list of $2749$ \scladj glitches in data from the LIGO-Hanford observatory and $1306$ from the LIGO-Livingston observatory.

The number of \scladj glitches found by the ArchEnemy pipeline can be compared to Gravity Spy for the same period of time. Gravity Spy finds $2731$ and $1396$ for LIGO-Hanford and LIGO-Livingston respectively~\cite{gravityspy:2023}. There will be a difference in the number of glitches found by ArchEnemy and Gravity Spy for at least two reasons: Gravity Spy treats an entire stack of harmonic glitches as a single \scladj glitch, whereas ArchEnemy will identify each glitch as a separate occurrence. Gravity Spy can also identify \scladj glitches which are not symmetric and fall outside our template bank, for example, the \scladj glitches shown in Figure~\ref{4:fig:overlay_bads}.

Figure~\ref{4:fig:overlay_goods} is an example of the results of the ArchEnemy pipeline and how well it has identified \scladj glitches in a period of data. A majority of the glitches have been identified with the correct parameter values and even in cases where the chosen template is not visually perfect, there is a good match between the template and the identified power in the data, particularly in the case of slightly asymmetric glitches. Figure~\ref{4:fig:overlay_bads} demonstrates a period of time when the ArchEnemy pipeline has not fitted well the \scladj glitches in the data. The glitches at this time are improperly fit by the templates due to asymmetry of the morphology of the glitches and because some glitches are outside our template bank parameter range. However, we note that this is a very extreme period of \scladj glitching and immediately after this time the detector data is no longer flagged as ``suitable for analysis''.

\begin{figure}
       \centering
    \includegraphics[width=0.7\linewidth]{images/4_archenemy/Section3/3.7/overlay_bad_overlays.pdf}
    \caption{LIGO-Livingston data from 2019-11-24 01:30:32--2019-11-24 01:31:20 containing a very large number of \scladj glitches at multiple times and frequencies over-plotted with the \scladj glitches identified by the ArchEnemy search. Very few overlays match well onto \scladj glitches. The template bank used in this search terminates at $80 \, \text{Hz}$ and so the \scladj glitches located above this value will not be correctly identified. There are also asymmetric \scladj glitches located which will not be identified correctly by our search, which assumes symmetry in the \scladj glitch.}
    \label{4:fig:overlay_bads}
\end{figure}

We have demonstrated the ArchEnemy pipeline on a stretch of data from O3 and have identified and characterised a list of \scladj glitches, which could be removed from the data. We do note that there are cases where the identification has not worked well, but we expect that subtracting our list of glitches from the data will reduce their effect on the \gwadj search. In the next section, we will demonstrate this by quantifying sensitivity with the PyCBC pipeline.

\subsection{Safety of \scl identification}
\label{4:ssec:injsafety}

The data we have searched through contains no previously identified \gwadj signals~\cite{gwtc3:2023}. However, there is a risk that the ArchEnemy search would identify real \gwadj signals as \scladj glitches. To assess this possibility, we simulate and add many \gwadj signals into the data and assess whether any signals are misidentified.

To do this, we use three separate sets of simulated \gwadj signals (or ``injection sets''), one for BBHs, another for binary neutron stars (BNSs) and a third for neutron star black hole (NSBH) systems. We use the same simulations as the LVK search of this data, detailed in the appendix of~\cite{gwtc3:2023}. Each injection set consists of $6200$ simulated signals spaced between $82$ and $120$ seconds apart. We treat these injection sets exactly the same as for the injection-less data, adding the simulations to the data, and then running ArchEnemy to produce a list of \scladj glitches for each injection set.

To determine whether we have misidentified any \gwadj injections as \scladj glitches we look for glitches we have found within the overlapping frequency band of \gwadj signals and our \scladj glitch template bank. This corresponds to approximately $15 \, \text{seconds}$ before merger time for the injections. The simulated signals occur every ${\sim}100$ seconds so we expect to see glitches within this $15$ second window, therefore, we also require that there must be more triggers identified in the \scladj glitch search \emph{with} injections when compared to the search \emph{without} injections within the window. The details of the number of \gwadj injections with overlapping \scladj triggers can be seen in Table~\ref{4:tab:coincident_triggers}.

\begin{table}[tb]
\caption{\label{4:tab:coincident_triggers}For both interferometers and all $3$ injection sets we identify the number of injections which are found to have \scladj glitches identified within $15$ seconds of merger time (``Injections with Coincident Triggers''), along with the number of \scladj glitches found within this window for these injections (``Scattered-Light Coincident Triggers''). We investigated each of these injections and recorded the number which actually had \scladj glitches identified due to the injected \gwadj signal (``Actual Overlapped Injections'').}
\footnotesize
\renewcommand{\arraystretch}{1.2}
\begin{tabular}{@{}cccccc}
\hline
&    & Injections with& Scattered-Light & Actual Overlapped \\
Interferometer & Injection Set & Coincident Triggers& Coincident Triggers&Injections \\
\hline
H1 & BBH           & 20 & 45 & 1\\
   & BNS           & 23 & 50 & 2\\
   & NSBH          & 38 & 73 & 7\\
L1 & BBH           & 13 & 21 & 2\\
   & BNS           & 18 & 30 & 2\\
   & NSBH          & 35 & 56 & 16\\
\hline
\end{tabular}

\end{table}

A \scladj glitch will be identified close to a \gwadj signal in two cases: the ArchEnemy search is misidentifying the \gwadj signal as a glitch \emph{or} the simulated signal was added close to actual glitches and a change in the data has meant a different number of glitches has been identified. The presence of real \scladj glitches means we might miss a \gwadj signal, therefore, we \emph{do} want to find and subtract glitches close to \gwadj signals, but we do not want to subtract power from the \gwadj signal itself. The \scladj glitch $\chi^{2}$ test was designed to prevent the matching of \scladj glitch templates to other causes of excess power, however, these results show it is not perfect.

We investigate each injection with coincident \scladj triggers, seeing how many had misidentified \scladj glitches on the inspiral of the \gwadj signal, this number can be seen in the column ``Actual Overlapped Injections'' in Table~\ref{4:tab:coincident_triggers}. We have included an example of the matching of \scladj glitches onto \gwadj injections in Figure~\ref{4:fig:loud_inj}, the right panel shows the \gwadj data post glitch subtraction where it can be seen there is a portion of the power being subtracted from the signal. Although power is being removed from the signal, the \gwadj injection is still found by the search for \gws, which we will describe later. For the cases that we have investigated, we note that the behaviour shown in Figure~\ref{4:fig:loud_inj} only happens for signals that have very large SNR, and are therefore unphysically close to us. A similar effect is observed with the ``autogating'' process, described in~\cite{PyCBC:2016}, which prevents the detection of these loud signals. In contrast to the ``autogating'' though, signals like that illustrated in Figure~\ref{4:fig:loud_inj} are still identified as \gwadj signals by the PyCBC search after \scladj glitch removal.

\begin{figure}
  \centering
  \begin{minipage}[t]{1.0\linewidth}
    \includegraphics[width=0.49\linewidth]{images/4_archenemy/Section3/3.8/L1_loud_Original.pdf}
    \hspace{0.02\linewidth}
    \includegraphics[width=0.49\linewidth]{images/4_archenemy/Section3/3.8/L1_loud_Subtracted.pdf}
  \end{minipage}
    \caption{An injected binary neutron star compact binary coalescence \gwadj signal, with the \scladj glitches identified by the ArchEnemy search pipeline overlaid in red (left). The same injected signal, but with the \scladj glitches removed from the data (right). It can be seen that power is removed from the signal track and also there is an amount of power added to the data above the track.}
    \label{4:fig:loud_inj}
\end{figure}

\section{\label{4:sec:results}Assessing sensitivity gain from removing \scladj glitches}

We now assess whether removing our identified list of \scladj glitches results in a sensitivity gain when searching for CBCs. We do this by comparing the results from the offline PyCBC search on the original data, to the results of the same search but analysing data where the glitches have been removed.

\subsection{Comparing search results with and without glitch subtraction}

The PyCBC pipeline is able to assess the significance of potential CBCs in a given stretch of data, and does the same with a set of simulated signals. This significance is quoted in terms of a ``false-alarm rate'', which denotes how often we would expect to see a non-astrophysical event at least as significant as the coincident trigger being considered. In this work, we assess sensitivity at a false-alarm threshold of 2 background events every year.

The data we have searched over contained no previously found \gwadj signals~\cite{gwtc3:2023} and our search after subtracting \scladj glitches identified no new \gwadj signals. While the search hasn't found any \gws, we can still measure the improvement in the sensitivity of the detectors by comparing the number of simulated signals identified with a false-alarm rate below 2 per year for each injection set (described in Section~\ref{4:ssec:injsafety}) with and without removing glitches from the data. Table \ref{4:tab:found_injs} shows the number of injections found for all injection sets and both searches.
%
\begin{table}[tb]
\centering
\caption{\label{4:tab:found_injs}The number of injections found by each search with a false-alarm rate less than 2 per year alongside the number of newly-found and newly-missed injections, those found by the glitch-subtracted and not the original search and vice versa. We also show the sensitivity ratio of the glitch-subtracted search and original search for each injection set.} 
\footnotesize
\renewcommand{\arraystretch}{1.2}
\begin{tabular}{@{}cccccc}
\hline
Injection & Original  & Glitch- & Sensitivity & Newly & Newly \\
Type & Search & Subtracted & ratio & Found & Missed \\
\hline
BBH & 1215 & 1222 & 1.01 & 10 & 3 \\
BNS & 1315 & 1315 & 1.00 & 5 & 5 \\
NSBH & 1260 & 1265 & 1.00 & 8 & 3 \\
\hline
\end{tabular}


\end{table}

We compare the number of injections found by both searches but also look at the \gwadj injections found by the original search and missed by the glitch-subtracted search and vice versa, this information can be seen in Table~\ref{4:tab:found_injs}. Considering signals found by the original search and missed by the glitch-subtracted search there are $3$ binary black hole injections with false-alarm rates in the original search ranging from $0.5\text{--}0.3$ per year, one of which had a glitch removed approximately $9$ seconds after the injection, there are $5$ newly-missed binary neutron star injections with false-alarm rates ranging from $0.5\text{--}0.056$ per year, two of the five binary neutron star injections had glitches removed within $60$ seconds of the injection, and there are $3$ newly-missed neutron star black hole injections with false-alarm rates ranging from $0.5\text{--}0.086$ per year, one had glitches removed within $60$ seconds of the injection. The other newly-missed injections showed no \scladj glitches within a $20$ second window for binary black hole injections and a $60$ second window for binary neutron star and neutron star black hole injections.

The glitch-subtracted search identifies $10$ additional binary black hole injections, the most significant of which have false-alarm rates of 1 per $190.50$, 1 per $7633.84$ and 1 per $8643.73$ years. We illustrate the last of these in Figure~\ref{4:fig:ae_found} (top). $5$ extra binary neutron star injections were found, with false-alarm rates from $0.5\text{--}0.19$ per year and $8$ neutron star black hole injections were found, where the false-alarm rate of the most significant is 1 per $9961.55$ years. This injection can also be seen in Figure~\ref{4:fig:ae_found} (bottom). We find $6$ of the $10$ binary black hole injections have \scladj glitches within a $20$ second window of the injection, $3$ of $5$ binary neutron star injections have \scladj glitches within a $60$ second window of the injection and, $6$ of the $8$ neutron star black hole injections have \scladj glitches within a $60$ second window of the injection. We provide more details about the newly-found and newly-missed injections in~\ref{4:sec:apdx_injections_table}

\begin{figure}
  \centering
  \begin{minipage}[t]{1.0\linewidth}
    \includegraphics[width=0.49\linewidth]{images/4_archenemy/Section4/4.1/BBH_H1_loud_Original.pdf}
    \hspace{0.02\linewidth}
    \includegraphics[width=0.49\linewidth]{images/4_archenemy/Section4/4.1/BBH_H1_loud_Subtracted.pdf}
  \end{minipage}
  \begin{minipage}[t]{1.0\linewidth}
    \includegraphics[width=0.49\linewidth]{images/4_archenemy/Section4/4.1/NSBH_H1_loud_Original.pdf}
    \hspace{0.02\linewidth}
    \includegraphics[width=0.49\linewidth]{images/4_archenemy/Section4/4.1/NSBH_H1_loud_Subtracted.pdf}
  \end{minipage}
    \caption{Two examples of \gwadj injections found by the glitch-subtracted search for \gws which were not found by the original \gwadj search due to the presence of \scladj glitches at the same time as the \gwadj inspiral. Top left: A binary black hole injection with a false-alarm rate of 1 per $8643.73$ years is shown alongside the \scladj glitches found by the ArchEnemy search and subtracted from the data prior to performing the glitch-subtracted PyCBC search for \gws (top right). Bottom left: A neutron star black hole injection with a false-alarm rate of 1 per $9961.55$ years and the \scladj glitches found by the ArchEnemy search and subtracted from the data prior to performing the glitch-subtracted PyCBC search for \gws (bottom right).}
    \label{4:fig:ae_found}
\end{figure}

To quantify the sensitivity of the search, we calculate the sensitive volume in which we can observe \gwadj signals. To calculate the sensitive volume we measure the detection efficiency of different distance bins taken from the injection sets and then multiply the efficiencies by the volume enclosed by the distance bins, these volumes are then summed to find the total volume the search is sensitive to~\cite{rw_snr_eq:2012}. We are then able to calculate the ratio in sensitivities between the glitch-subtracted \gwadj search and the original \gwadj search,  revealing the improvement that subtracting \scladj glitches has made.

Figure~\ref{4:fig:allinj_vt_ratio} displays the ratio of the sensitive volume measured for the glitch-subtracted \gwadj search and the original PyCBC \gwadj search, across different false-alarm rate values, we quote our sensitivity ratios at a false-alarm rate value of 2 per year. The same set of injected signals was used for both \gwadj searches, and therefore a direct comparison of search sensitivities can be made via this ratio. Disappointingly, the measured sensitivity improvement is small in the results we obtain. For the binary black hole injections we measure a sensitivity ratio at a 2 per year false-alarm rate of $1.01$, for binary neutron stars $1.00$ and neutron-star--black-holes $1.00$. 

The statistical significance of the sensitivity increase we report for the binary black hole injection set can be found by investigating the null hypothesis of seeing the same $1\%$ increase under the assumption that the subtraction of \scladj glitches does not actually increase sensitivity. When performing this analysis we find that our result is not statistically significant at the 95\% confidence interval --i.e. there is a 5.24\% chance that we would measure such an increase in sensitivity at least as large as this under the null hypothesis-- (see~\ref{4:sec:apdx_stat_sig} for details). However, the marginal sensitivity increase would not justify repeating the \scladj glitch search and glitch-subtracted \gwadj search on a larger injection set, instead more work is needed to better identify and remove \scladj glitches while remaining safe in the presence of \gwadj signals.

\begin{figure}
     \centering
     \includegraphics[width=0.7\textwidth]{images/4_archenemy/Section4/4.1/allinj_vt_ratio_ArchEnemy_coinc.pdf}
     \caption{The ratio of the sensitive volume-time of the glitch-subtracted search and the original \gwadj search. The grey dashed line indicates a false-alarm rate of $2$ per year, which is our threshold and the point at which we measure any sensitivity improvements of the glitch-subtracted search for each of the three \gwadj injection sets. }
     \label{4:fig:allinj_vt_ratio}
\end{figure}

\subsection{\label{4:sec:apdx_injections_table}Injections tables}

Here we have three tables which contain data on the newly-found and newly-missed injections for each injection set. The tables are separated into the values for the inverse false-alarm rate and ranking statistic found for each injection in both searches then the SNR, $\chi^{2}$ and PSD variation values for each detector in both searches. The first table is the results of the binary black hole injection set, the second table is the binary neutron star injection set and the third is the neutron star black hole injection set. A horizontal dashed line separates newly-found and newly-missed injections, using a false-alarm rate threshold of 2 per year. There were 2 newly-found binary black hole injections which were not found at all by the original search, and therefore they do not appear in the first table.

There have been 34 newly-found or newly-missed \gwadj injections when subtracting \scladj glitches, it is informative to understand how the \gwadj search is influenced by the glitch subtraction to cause this outcome. The ranking statistic~\cite{PyCBC_global:2020} represents the legitimacy of a signal being astrophysical in origin and is partially computed using the re-weighted SNR, which is itself computed using the initial SNR alongside the various \gwadj discriminators and the PSD variation measurement~\cite{PSD_var:2020}. We use the trigger information saved by the \gwadj search to identify why injections that weren't found previously have been found post-glitch subtraction, and vice versa. 

As an example, we take the smallest false-alarm rate (1 per $8643.73$ years), newly-found, binary black hole injection, and look at the ranking statistic, SNR, $\chi^{2}$ and, PSD variation measurements in both detectors and both searches -- these values can be found in Table~\ref{4:tab:apdx_changed_snr_bbh}. This injection was originally seen with a false-alarm rate of 100 per year, far above our threshold, a very large increase in the ranking statistic from 13.29 to 27.01 is certainly responsible for the decreased false-alarm rate. There were no changes in the values measured by the LIGO-Livingston detector between searches which is expected as no \scladj glitches were found within $512$ seconds of the injection. The LIGO-Hanford detector sees a small increase in the SNR measured, from 7.98 to 8.22, a small increase in the $\chi^{2}$ value, from 2.32 to 2.48, and a very significant decrease in the PSD variation measurement, from 3.47 to 1.50. Using equation 18 of~\cite{PSD_var:2020}, we can calculate a re-weighted SNR of $4.60$ for the original search and $5.67$ for the glitch-subtracted search, a significant increase in the SNR. Similar analyses for all the newly-found and newly-missed injections can be made using information found in tables~\ref{4:tab:apdx_changed_snr_bns} and~\ref{4:tab:apdx_changed_snr_nsbh}.

The decrease in the PSD variation is true for the three newly-found very low false-alarm rate injections, accompanied by the small changes in SNR and increase in the $\chi^{2}$ measurement. For newly-found and newly-missed injections which lie close to the 2 per year false-alarm rate threshold, there is no definitive reason as to why these injections changed state.

\newpage
\newgeometry{left=2cm,right=2cm,top=2cm,bottom=2cm} % set new margins for one page
\begin{landscape}
\begin{table}[tb]
\centering
\caption{\label{4:tab:apdx_changed_snr_bbh}This table contains the trigger information for the newly-found and newly-missed \textbf{binary black hole injections} recorded by the original search~\cite{gwtc3:2023} and the glitch-subtracted search.} 
\begin{tabular}{|c|c|c|c|c|c|c|c||c|c|c|c|c|c|c|c|}
\hline
\multicolumn{8}{|c||}{Glitch-Subtracted} & \multicolumn{8}{c|}{Original Search} \\
\hline
\multicolumn{2}{|c|}{} & \multicolumn{3}{c|}{H1} & \multicolumn{3}{c||}{L1} & \multicolumn{2}{c|}{} & \multicolumn{3}{c|}{H1} & \multicolumn{3}{c|}{L1}\\
\hline
IFAR & Ranking & SNR & $\chi^{2}$ & PSD & SNR & $\chi^{2}$ & PSD & IFAR & Ranking & SNR & $\chi^{2}$ & PSD & SNR & $\chi^{2}$ & PSD \\ &

Stat. & & & Var. & & & Var. & & Stat. & & & Var. & & & Var.\\
\hline
8643.73 & 27.01 & 8.22 & 2.48 & 1.50 & 8.05 & 2.07 & 1.00 & 0.01 & 13.29 & 7.98 & 2.32 & 3.47 & 8.05 & 2.07 & 1.00 \\
7633.84 & 26.13 & 8.18 & 2.25 & 1.22 & 7.58 & 1.46 & 1.03 & 0.37 & 16.63 & 8.19 & 1.87 & 2.17 & 7.59 & 1.45 & 1.02 \\
4.89 & 19.06 & 7.23 & 1.88 & 0.97 & 10.34 & 3.15 & 1.05 & 1.56 & 18.01 & 7.24 & 2.11 & 0.97 & 10.34 & 3.42 & 1.04 \\
3.14 & 18.65 & 5.89 & 1.54 & 1.01 & 8.86 & 2.40 & 1.09 & 1.52 & 17.99 & 5.89 & 1.54 & 1.01 & 8.86 & 2.40 & 1.09 \\
2.75 & 18.53 & 8.50 & 1.63 & 0.99 & 6.80 & 1.37 & 1.05 & 1.51 & 17.98 & 8.51 & 1.56 & 0.98 & 6.80 & 1.37 & 1.05 \\
2.39 & 18.41 & 7.83 & 2.00 & 1.01 & 8.41 & 3.41 & 1.02 & 1.06 & 17.65 & 7.81 & 2.14 & 1.01 & 8.41 & 3.41 & 1.02 \\
2.18 & 18.32 & 7.02 & 1.90 & 1.02 & 6.35 & 2.15 & 1.01 & 1.79 & 18.14 & 7.05 & 1.92 & 1.02 & 6.35 & 2.16 & 1.01 \\
2.01 & 14.11 & 7.94 & 1.89 & 1.01 & 5.67 & 1.95 & 1.16 & 1.90 & 14.14 & 7.94 & 1.90 & 1.01 & 5.07 & 0.00 & 1.02 \\
\hdashline
1.76 & 18.12 & 6.89 & 1.43 & 1.02 & 6.76 & 2.59 & 0.97 & 3.33 & 18.71 & 6.86 & 1.82 & 1.02 & 6.68 & 2.53 & 0.97 \\
1.79 & 18.14 & 6.40 & 1.69 & 1.02 & 7.56 & 2.01 & 1.01 & 2.05 & 18.27 & 6.40 & 1.69 & 1.02 & 7.56 & 2.00 & 1.01 \\
1.65 & 18.06 & 5.14 & 0.00 & 0.91 & 7.57 & 1.57 & 1.02 & 2.05 & 18.27 & 5.14 & 0.00 & 0.91 & 7.57 & 1.57 & 1.02 \\
\hline
\end{tabular}
\end{table}
\end{landscape}
\restoregeometry % restore the default margins for subsequent pages

\newpage

\newgeometry{left=1cm,right=1cm,top=2cm,bottom=2cm} % set new margins for one page
\begin{landscape}
\begin{table}[tb]
\centering
\caption{\label{4:tab:apdx_changed_snr_bns}This table contains the trigger information for the newly-found and newly-missed \textbf{binary neutron star injections} recorded by the original search~\cite{gwtc3:2023} and the glitch-subtracted search.} 
\begin{tabular}{|c|c|c|c|c|c|c|c||c|c|c|c|c|c|c|c|}
\hline
\multicolumn{8}{|c||}{Glitch-Subtracted} & \multicolumn{8}{c|}{Original Search} \\
\hline
\multicolumn{2}{|c|}{} & \multicolumn{3}{c|}{H1} & \multicolumn{3}{c||}{L1} & \multicolumn{2}{c|}{} & \multicolumn{3}{c|}{H1} & \multicolumn{3}{c|}{L1}\\
\hline
IFAR & Ranking & SNR & $\chi^{2}$ & PSD & SNR & $\chi^{2}$ & PSD & IFAR & Ranking & SNR & $\chi^{2}$ & PSD & SNR & $\chi^{2}$ & PSD \\ &
Stat. & & & Var. & & & Var. & & Stat. & & & Var. & & & Var.\\
\hline
5.12 & 19.12 & 5.86 & 1.99 & 1.01 & 7.15 & 1.94 & 1.02 & 1.49 & 17.96 & 5.74 & 2.14 & 1.01 & 7.15 & 1.94 & 1.02 \\
4.91 & 19.07 & 5.48 & 2.29 & 1.01 & 7.92 & 2.16 & 1.01 & 1.37 & 17.89 & 5.78 & 2.25 & 1.01 & 7.23 & 2.13 & 1.01 \\
4.54 & 18.99 & 4.82 & 0.00 & 1.07 & 8.68 & 2.05 & 1.06 & 1.66 & 18.07 & 4.82 & 0.00 & 1.14 & 8.68 & 2.05 & 1.06 \\
2.46 & 18.43 & 5.48 & 1.95 & 1.05 & 7.61 & 2.03 & 0.99 & 1.78 & 18.13 & 5.48 & 1.94 & 1.05 & 7.62 & 2.08 & 0.99 \\
2.34 & 18.39 & 6.25 & 2.18 & 1.01 & 6.82 & 1.96 & 1.00 & 1.45 & 17.94 & 6.09 & 2.05 & 1.01 & 6.83 & 2.01 & 1.00 \\
\hdashline
1.24 & 17.71 & 6.58 & 2.14 & 1.01 & 6.89 & 2.48 & 1.01 & 17.81 & 20.19 & 6.58 & 2.15 & 1.01 & 7.16 & 2.26 & 1.01 \\
1.14 & 17.72 & 6.00 & 2.67 & 0.98 & 8.07 & 2.66 & 1.22 & 7.14 & 19.44 & 6.17 & 2.45 & 0.97 & 8.26 & 2.80 & 1.22 \\
1.89 & 18.19 & 6.32 & 2.19 & 1.00 & 6.79 & 1.83 & 0.99 & 3.57 & 18.78 & 6.37 & 2.12 & 0.99 & 6.79 & 1.75 & 0.99 \\
0.88 & 17.46 & 6.88 & 2.14 & 0.99 & 6.19 & 2.12 & 0.99 & 2.33 & 18.39 & 6.87 & 2.07 & 0.99 & 6.19 & 1.99 & 0.99 \\
1.36 & 17.80 & 6.20 & 2.02 & 1.00 & 6.83 & 2.35 & 1.00 & 2.07 & 18.18 & 6.20 & 2.02 & 1.00 & 6.83 & 2.28 & 1.00 \\
\hline
\end{tabular}
\end{table}
\end{landscape}
\restoregeometry % restore the default margins for subsequent pages

\newpage

\newgeometry{left=1cm,right=1cm,top=2cm,bottom=2cm} % set new margins for one page
\begin{landscape}
\begin{table}[tb]
\centering
\caption{\label{4:tab:apdx_changed_snr_nsbh}This table contains the trigger information for the newly-found and newly-missed \textbf{neutron star black hole injections} recorded by the original search~\cite{gwtc3:2023} and the glitch-subtracted search.} 
\begin{tabular}{|c|c|c|c|c|c|c|c||c|c|c|c|c|c|c|c|}
\hline
\multicolumn{8}{|c||}{Glitch-Subtracted} & \multicolumn{8}{c|}{Original Search} \\
\hline
\multicolumn{2}{|c|}{} & \multicolumn{3}{c|}{H1} & \multicolumn{3}{c||}{L1} & \multicolumn{2}{c|}{} & \multicolumn{3}{c|}{H1} & \multicolumn{3}{c|}{L1}\\
\hline
IFAR & Ranking & SNR & $\chi^{2}$ & PSD & SNR & $\chi^{2}$ & PSD & IFAR & Ranking & SNR & $\chi^{2}$ & PSD & SNR & $\chi^{2}$ & PSD \\ &
Stat. & & & Var. & & & Var. & & Stat. & & & Var. & & & Var.\\
\hline
9961.55 & 27.41 & 5.55 & 2.08 & 1.23 & 10.26 & 2.27 & 1.12 & 0.01 & 12.83 & 6.01 & 2.39 & 2.25 & 8.54 & 2.37 & 1.12 \\
141.35 & 22.28 & 7.76 & 2.20 & 1.02 & 6.54 & 2.35 & 1.01 & 1.21 & 17.77 & 7.83 & 1.94 & 1.02 & 7.29 & 2.50 & 1.00 \\
18.26 & 20.36 & 5.94 & 2.20 & 1.09 & 6.02 & 1.99 & 1.10 & 1.29 & 17.82 & 5.74 & 2.05 & 1.09 & 6.27 & 2.10 & 1.18 \\
17.12 & 20.29 & 7.64 & 2.00 & 1.11 & 6.66 & 2.15 & 1.05 & 1.37 & 17.89 & 7.36 & 2.35 & 1.11 & 6.66 & 2.15 & 1.05 \\
12.94 & 20.03 & 6.52 & 1.79 & 1.07 & 7.31 & 2.32 & 1.13 & 1.96 & 18.23 & 6.57 & 1.85 & 1.20 & 7.31 & 2.32 & 1.13 \\
7.12 & 19.42 & 5.13 & 0.00 & 1.05 & 8.33 & 2.08 & 1.04 & 0.03 & 14.23 & 5.30 & 2.16 & 1.05 & 7.44 & 2.43 & 1.04 \\
3.61 & 18.77 & 5.80 & 2.01 & 1.02 & 7.43 & 1.88 & 0.93 & 1.81 & 18.15 & 5.80 & 2.01 & 1.02 & 7.43 & 1.97 & 0.93 \\
2.15 & 18.22 & 7.41 & 2.04 & 1.02 & 5.57 & 2.03 & 1.03 & 0.00 & 4.34 & 5.64 & 2.18 & 0.98 & 5.04 & -0.00 & 1.01 \\
\hdashline
0.23 & 16.16 & 5.79 & 1.81 & 0.96 & 8.78 & 3.93 & 1.03 & 11.69 & 19.93 & 5.78 & 1.89 & 0.96 & 8.79 & 3.17 & 1.03 \\
1.56 & 17.92 & 6.71 & 1.91 & 0.99 & 6.73 & 2.50 & 1.05 & 2.68 & 18.41 & 6.71 & 1.90 & 0.99 & 6.72 & 2.39 & 1.05 \\
1.72 & 18.10 & 5.77 & 1.90 & 0.99 & 6.77 & 1.74 & 1.00 & 2.08 & 18.28 & 5.77 & 1.90 & 0.99 & 6.80 & 1.69 & 1.00 \\
\hline
\end{tabular}
\end{table}
\end{landscape}
\restoregeometry % restore the default margins for subsequent pages

\section{\label{4:sec:apdx_stat_sig}Statistical significance}

We report a $1\%$ increase in the sensitivity for the binary black hole injection set in the glitch-subtracted \gwadj search (see Section~\ref{4:sec:results}). We wish to determine the statistical significance of this result under the null hypothesis that subtracting \scladj glitches prior to searching for \gws does not increase the sensitivity of the \gwadj search. The binary black hole injection set contains $6200$ injected signals, $10$ additional injections were found in the glitch-subtracted \gwadj search, $3$ additional injections were missed in our search, this gives us $13$ injections which have changed state.

First, we calculate the probability that an injection has been affected by the glitch removal, 
%
\begin{equation}
    p = \frac{13}{6200} = 0.21\% ,
\end{equation}
%
then we calculate the standard deviation, 
%
\begin{equation}
    \textrm{std} = \sqrt{n * p * (1 - p)} = 3.60 .
\end{equation}
%
Using the standard deviation we calculate the number of standard deviations our result deviates from the mean. We divide the number of positively changed (newly-found) injections by the standard deviation, 
%
\begin{equation}
    \textrm{standard deviation} = \frac{(10 - 3)}{3.60} = 1.94 .
\end{equation}
%
Under the assumption of no sensitivity increase caused by the subtraction of glitches, we measure our result of +7 newly-found \gwadj injections to lie $1.94$ standard deviations from an expected value of $0$ newly-found injections. The critical value of a 95\% confidence interval, that is to say there is a 1 in 20 chance of our null hypothesis being true, is $1.96$ meaning our result is within the 95\% confidence interval. We can describe this as there being a 5.24\% chance that our result is not caused by the subtraction of glitches but is instead caused by random chance. To reduce the error in the computed sensitivity ratio a larger injection set test would be required, this would need a large time and computational power investment which we do not believe is justified in the case of such a minor increase in the sensitivity.

\section{\label{4:sec:conclusion}Conclusion}

We have demonstrated a new method for modelling \scladj glitches and identifying and characterizing these glitches in a period of \gwadj data. We have developed a \scladj glitch specific $\chi^{2}$ test which can discriminate between \scladj glitches, other types of glitches and \gwadj signals. We have searched through a representative stretch of \gwadj data known to contain \scladj glitches, found thousands of these glitches and subtracted them from the \gwadj data prior to running a search for \gws. The results of this search include a small increase in the measured sensitivity of the \gwadj search for binary black hole \gwadj signals, and modest change to sensitivity for binary neutron star and neutron star black hole \gwadj signals.

We highlight that the task of accurately identifying and parameterizing \scladj glitches in the data is not a trivial one, especially where there are repeated, and harmonic, glitches present in the data. We have developed a new $\chi^{2}$ test to reduce the number of false identifications of \scladj glitches, but we do still see cases where we have misidentified other glitches, and even some loud \gwadj signals, as caused by \scl, and cases where we do not correctly identify, or parameterize, actual \scladj glitches. Improving this identification process would be important in improving the efficacy of this process.

The possibility of using this model of \scladj glitches as a bespoke application to \gwadj signals which are known to have coincident \scladj glitches has been explored and implemented into Bilby~\cite{BILBY:2019} to perform a parameter estimation of \scladj glitches and removing these to produce glitch-free data~\cite{Udall:2023}. The inclusion of the extra term from~\cite{Was_Subtract:2021} within the model can help identify \scladj glitches more accurately. Selectively subtracting glitches based on the presence of a \gwadj signal is possible by moving the glitch subtraction process inside the \gwadj searches and including the results of other \gwadj discriminators~\cite{rw_snr_eq:2012, McIsaac_Chi:2022} to determine the legitimacy of an ArchEnemy identified glitch.

The results of the application of the ArchEnemy search pipeline, the list of \scladj glitches, can also be used in other applications. For example, it could be used in the form of a veto~\cite{O2O3_DetChar:2021}, where we use knowledge of the presence of \scladj glitches to down rank periods of time in \gwadj data. Additionally, we could use \scladj glitches previously identified by tools such as Gravity Spy~\cite{gravityspy:2023} and target known \scladj glitches with the ArchEnemy search pipeline.

As a final note, while we acknowledge that the sensitivity improvements that we have observed---${\sim} 1\%$---are very modest, the concept of removing \scladj glitches, or other identified glitch classes, from the data prior to matched filtering for CBCs is one that we encourage others to explore further. An increase in the rate of events or the rate of \scladj glitches in future observing runs will mean an increase in the number of affected events, such techniques offer a method for mitigating the effect that these glitches will have on the search, maximizing the number of observations that can be made.
